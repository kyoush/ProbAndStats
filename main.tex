\documentclass[autodetect-engine]{jsarticle}

\usepackage[sc]{mathpazo}
\usepackage{newpxtext}
%枠で囲む
\usepackage{ascmac}
\usepackage{amsmath}

\title{Prob \& Stats}
\author{Kyoshiro Kamura}
\date{\today}

\begin{document}
\maketitle

{\fontsize{10.5bp}{16.8bp}\selectfont
  \section{Introduction}
  Prob \& Stats is Not New.
  
  The concept of chance of uncertainty.
  
  \section{Sets and Elements}
  A \textbf{``set''} is clealy defined collection of \textbf{``elements''}.

  Sets are clealy defined, but not nessesarily finite.

  \subsection{Symbols}
  When element ``$a$'' belongs to set ``$A$'', we write
  \[
  a \in A
  \]
  That means ``$a$ is an element of $A$.''.

  If element ``$b$ is outside set ``$A$'', we write
  \[
  b \notin A
  \]

  If the elements of set $B$ are contained within set $A$
  \[
  B \subset A
  \]
  That means ``$B$ is subset of $A$.''.

  A set containing no elements is called \textbf{``null set''}
  and witten as "$\emptyset$''.

  \subsection*{\underline{Let's consider 2 sets $A$ and $B$}}
  The set created by the elements belonging to both $A$ \& $B$ is called the \textbf{``intersection''}.
  \[
  A \cap B
  \]

  The set created by all the element of $A$ \& $B$ is called the \textbf{``union''}
  \[
  A \cup B
  \]

  All elements outside both $A$ \& $B$ from the ``complementary'' set.
  \[
  \overline{A \cup B} = U - (A \cup B)
  \]
  \begin{itembox}[l]{Note}
    In some textbook, this symbol, ``$U$'' means \textbf{``Everything''}.
  \end{itembox}

  \subsubsection*{\underline{Practice Excercise}}
  \[
  U = \{1, 2, 4, 8, 10\}, A = \{4, 8\}
  \]
  \begin{flushright}
    Note : ``$U$'' is the ``entire universe'', and $A$ is a subset of $U$.
  \end{flushright}

  \begin{flalign*}
    \displaystyle
    &1.\;A \cap U = \{4, 8\}&\\
    &2.\;A \cup U = \{1, 2, 4, 8, 10\}&\\
    &3.\;B = \overline{A} = \{1, 2, 10\}&\\
    &4,\;A \cap B = \emptyset&\\
    &5,\;A \cup B = \{1, 2, 4, 8, 10\}&
  \end{flalign*}

  \subsection{Rules}

}
\end{document}
